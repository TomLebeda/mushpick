\author{Ing.~Tomáš Lebeda}

% \usepackage{lmodern} % smooth font with accents like ěščřéíáý
\usepackage{csquotes} % czech styled quotes
\usepackage[czech]{babel} % czech babel
\usepackage{mathtools} % math magic
\usepackage{amssymb} % some funky math symbols like \trinangleq
\AtBeginDocument{\shorthandoff{-}} % czech babel does some black magic with dashes and it breaks other packages, this should disable it
\usepackage{geometry}
\usepackage[indent=0pt]{parskip} % spaces between paragraphs, indents sets indent of the first line in paragraph
% \usepackage{hyperref}
\usepackage{pdflscape}
\usepackage{enumitem}
\usepackage{caption}
\usepackage{subcaption}
\usepackage{fontspec}
\usepackage[warnings-off={mathtools-colon,mathtools-overbracket}]{unicode-math}
\usepackage{fancyvrb}
\usepackage{xcolor}
\usepackage{listings}
\usepackage{silence} % to filter out some warning and errors after compilation
\WarningFilter*{latex}{Text page \thepage\space contains only floats}
\ActivateWarningFilters
\setmainfont[
	Extension=.otf,
	UprightFont={*-Regular},
	BoldFont={*-Bold},
	ItalicFont={*-Italic},
	BoldItalicFont={*-BoldItalic},
	% RawFeature={+zero}
]{LibertinusSerif}
\setmathfont{LibertinusMath-Regular.otf}

\defaultfontfeatures{
	Ligatures={NoCommon}
}
% \setmonofont{IosevkaSlabMono}[
% 	Extension=.ttf,
% 	Scale=MatchLowercase,
% 	% Path=/home/tom/.local/share/fonts/,
% 	Path={/home/tom/Tools/iosevka-generate/FontPatcher/},
% 	UprightFont={*-Regular},
% 	BoldFont={*-Heavy},
% 	ItalicFont={*-Italic},
% 	BoldItalicFont = {*-ExtraBoldItalic},
% 	Ligatures={NoCommon}
% ]

\usepackage{tcolorbox} % pretty colored boxes
\tcbuselibrary{breakable} % make the colored boxes breakable across pages
\tcbuselibrary{skins} % make the color box breaks nicer

\usepackage{tikz} % pretty pictures with magic code
\usetikzlibrary{arrows.meta} % for the pretty arrowhead {Stealth[scale=1.0]}
\usetikzlibrary{shapes.geometric} % for ellipses and other shapes for nodes
\usetikzlibrary{calc} % for coordinate calculations
\usetikzlibrary{angles} % for easy angle display
% FPU for better floating point accuracy is at the end of this file

\graphicspath{{src/}}

\usepackage{pgfplots}
% \usepgfplotslibrary{external}
% \tikzset{external/system call={lualatex \tikzexternalcheckshellescape -halt-on-error -interaction=batchmode -jobname "\image" "\texsource"}}
% \tikzexternalize[prefix=tikz/]
\pgfplotsset{compat=1.18}
\pgfplotsset{table/search path={src}} % so autocomplete can suggest the files
\pgfplotsset{
	every tick label/.append style={
			font=\scriptsize
		},
	every axis label/.append style={
			font=\scriptsize
		},
	every axis legend/.append style={
			font=\scriptsize,
			legend cell align = left,
		},
	every legend image post/.append style = {
			line width=2pt
		},
	every axis/.append style = {
			grid = both,
			enlarge x limits = {0.02, auto},
			width=12cm,
			height=8cm
		}
}

\renewcommand{\baselinestretch}{1.3} % 1.3 odpovídá řádkování 1.5 (viz: https://latexref.xyz/_005cbaselineskip-_0026-_005cbaselinestretch.html)
\clubpenalty=10000 % remove clubs
\widowpenalty=10000 % remove widows

% Syntax: [draw options] (center) (initial angle:final angle:radius)
% You must add semicolon at the end! (don't add it here because this way I can chain another commands like \node after it)
% don't use is after "draw", it has it's own "draw" command!
% texlab: ignore
\def\centerarc[#1](#2)(#3:#4:#5){ \draw[#1] ($(#2)+({#5*cos(#3)},{#5*sin(#3)})$) arc (#3:#4:#5) }

\newcommand{\BrC}[1]{\left\lbrace\, #1 \,\right\rbrace}
\newcommand{\BrS}[1]{\left[\, #1 \,\right]}
\newcommand{\Br}[1]{\left(\, #1 \,\right)}
\newcommand{\BrA}[1]{\left\langle\, #1 \,\right\rangle}
\newcommand{\Abs}[1]{\left\vert #1 \right\vert}
\newcommand{\Norm}[1]{\left\| #1 \right\|}
\newcommand{\E}[1]{\text{E}\left[ #1 \right]}
\newcommand{\var}[1]{\text{var}\left[ #1 \right]}
\newcommand{\cov}[1]{\text{cov}\left[ #1 \right]}
\newcommand{\parfrac}[2]{\frac{\partial #1}{\partial #2}}
\newcommand{\musteq}{\stackrel{!}{=}}
\newcommand{\To}{\Rightarrow}
\newcommand{\TO}{\Rightarrow}
\newcommand{\celsius}{{}^\circ \text{C}}

\newcounter{example}[section]
\newenvironment{example}
{
	\stepcounter{example}
	\begin{tcolorbox}
		[
			enhanced jigsaw,
			breakable,
			parbox=false,
			title=Příklad \arabic{chapter}.\arabic{section}.\arabic{example},
			colback=white,
			colframe=blue!20!white,
			fonttitle=\bfseries,
			coltitle=black]
		}
		{
	\end{tcolorbox}
}

\newenvironment{definition}[1]
{
	\begin{tcolorbox}
		[
			parbox=false,
			title={\textit{Definice}: #1},
			colback=red!3!white,
			colframe=red!20!white,
			fonttitle=\bfseries,
			coltitle=black]
		}
		{
	\end{tcolorbox}
}

\let\verbatim\Verbatim
\let\endverbatim\endVerbatim

%last - to be sure that nothing will overwrite it
\usetikzlibrary{fpu}
